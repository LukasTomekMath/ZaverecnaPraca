% ============== PREAMBULA ==============
% (nacitanie balikov a podobne)
% Ctrl+klik na balik zobrazi dokumentaciu k baliku (v TeXstudiu)
% Je tu vela uzitocnych balikov ak ale nejaky nepotrebujete, zakomentujte ho

%definovanie verzie dokumentu (pdf/tlacena)
\usepackage{etoolbox}
\newbool{printVersion} % premenna pre volbu verzie
%\booltrue{printVersion} % tlacena: odkomentovat, pdf: nechat zakomentovanie


% geometria a formatovanie stran
\pagestyle{plain} % defaultny styl stran (plain/headings/empty/myheadings)
\ifbool{printVersion}{
	\usepackage[top=2.5cm, bottom=2.5cm, left=3.5cm, right=2cm]{geometry} % odporucane okraje
}{
	\usepackage[top=2.5cm, bottom=2.5cm, left=2.75cm, right=2.75cm]{geometry} % okraje vhodnejsie do pdf (rovnaka sirka vlavo/vpravo) 
}

% jazyk
\usepackage[main=slovak,english]{babel}  % primarny jazyk: SK, dalsie jazyky: EN


% bibliografia
\usepackage[style=iso-numeric,backend=bibtex,giveninits=true,sorting=nyt]{biblatex}
\addbibresource{literatura.bib}


% fromatovanie
\usepackage{indentfirst} % odsadenie prveho odstavca
\usepackage{enumitem} % číslovanie ((a),(b),(c),... i), ii), iii),...)
\usepackage[small]{caption} % male popisy obrazkov


% grafika
\usepackage{graphicx} % obrazky
\usepackage{subcaption} % podobrazky (subfigure)
\graphicspath{ {./figures/} } % priecinok s obrazkami
\usepackage{wrapfig} % obrazky obtekane textom
\usepackage[dvipsnames]{xcolor} % farby (dvipsnames pridava dalsie farby)
\usepackage{colortbl} % farebne tabulky
\usepackage{multirow} % treba pre merge-ovanie budniek v tabulkach

\usepackage{pdfpages} % vlozenie stran z ineho pdf (musi byt za graphics)


% matematika
\usepackage{mathtools} % rozsirenie zakladneho matematickeho balika amsmath
\usepackage{amssymb} % dalsie symboly a fonty (napriklad pre mnozinu realnych cisel)
\usepackage[locale=DE]{siunitx} % jednotky SI

% Definície, Vety, Dôkazy...
\usepackage{amsthm}
\theoremstyle{definition}
\newtheorem{thm}{Veta}[chapter]
\newtheorem{defn}[thm]{Definícia}
\newtheorem{lem}[thm]{Lemma}
\newtheorem{cor}[thm]{Dôsledok}
\newtheorem{rem}[thm]{Poznámka}
\newtheorem{exmp}[thm]{Príklad}


% pseudokod a kod
\usepackage[czech]{algorithm2e} % pseudokody/algoritmy
\usepackage{listings} % vkladanie kodu


% hyperreferencia
\ifbool{printVersion}{
	% hyperreferencia v tlacenej verzii
	\usepackage[hidelinks]{hyperref} % hyperreferencia
}{
	% hyperreferencia v pdf verzii
	\usepackage{hyperref} % hyperreferencia
	\hypersetup{colorlinks, % farby linkov
		citecolor = red,
		linkcolor = blue,
		urlcolor = blue}
}


% Ked uz je praca dlha, pre urychlenie kompilacie sa oplati vkladat len niektore kapitoly
%\includeonly{
	%%	Uvod,
	%%	Kapitola1,
	%%	Kapitola2,
	%	Kapitola3,
	%%	Zaver
	%}
