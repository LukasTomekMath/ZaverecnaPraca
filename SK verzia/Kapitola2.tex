\chapter{Členenie práce}

Keď začínate prácu písať, je dobré urobiť si základnú kostru -- štruktúru kapitol, podkapitol,... Premyslieť si a napísať krátku poznámku, o čom sa v časti bude hovoriť a až potom začať jednotlivé časti postupne napĺňať. Samozrejme, členenie sa časom bude upravovať, ale je dôležité mať na začiatku nejakú predstavu.

Okrem základného členenia (\texttt{\textbackslash chapter}, \texttt{\textbackslash section}, \texttt{\textbackslash subsection},...) sa v špecifických prípadoch môže hodiť rozdelenie na časti pomocou \texttt{\textbackslash part} (o úroveň nad \texttt{\textbackslash chapter}).


\section{Podkapitola (sekcia)}\label{}
Blabla

\subsection{Subsekcia}
Blabla

\subsubsection{Subsubsekcia}
Blabla

%\paragraph{Paragraf}
%Blabla
%
%\subparagraph{Subparagraf}
%Blabla