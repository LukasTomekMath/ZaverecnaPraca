\ifbool{printVersion}{
	\cleardoublepage % ak to vychadza na parnu stranu, vlozi sa prazdna strana
}

\section*{Abstrakt}

\noindent \textbf{Názov práce:} Názov práce v slovenčine\\
\textbf{Abstrakt:} Abstrakt má byť krátky, presný a zrozumiteľný text, ktorým predstavujete svoju prácu. Píše sa v prítomnom čase ako jeden odsek a mal by mať zhruba 50 -- 300 slov. Najlepšie je písať ho nakoniec, keď už má človek všetko dobre uležané v hlave a venovať mu dostatočnú pozornosť. Pri odovzdaní práce sa slovenský aj anglický Abstrakt zadáva aj do AIS a dá sa chápať ako \uv{upútavka} na vašu prácu.

\vspace{10pt}

\noindent \textbf{Kľúčové slová:} 3 až 5 kľúčových slov/slovných spojení oddelených čiarkou

\vspace{+20pt}


\section*{Abstract}
\noindent \textbf{Title:} Názov práce v angličtine\\
\textbf{Abstract:} Preklad abstraktu do angličtiny (poriadny, nie Google Translate). Je dobré ho robiť až keď je autor (a školiteľ) spokojný s tým slovenským, aby ste ho zbytočne nemuseli prekladať niekoľkokrát.

\vspace{10pt}

\noindent \textbf{Keywords:} 3 až 5 kľúčových slov/slovných spojení oddelených čiarkou
