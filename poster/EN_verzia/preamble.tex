% ============== PREAMBULA ==============
% (nacitanie balikov a podobne)
% Ctrl+klik na balik zobrazi dokumentaciu k baliku (v TeXstudiu)
% Je tu vela uzitocnych balikov ak ale nejaky nepotrebujete, zakomentujte ho


% geometria a formatovanie
\usepackage[left=2cm,right=2cm,top=0cm,bottom=0cm, paperwidth=700mm, paperheight=1000mm]{geometry}
%\usepackage{a0poster}
\usepackage{multicol}
 
\columnseprule=2pt % Hrubka oddelovaca medzi stlpcami
\columnsep=100pt % Medzera medzi oddelovacmi stlpcov a textom


% jazyk
\usepackage[main=english,slovak]{babel}  % primarny jazyk: EN, dalsie jazyky: SK


% bibliografia
\usepackage[style=iso-numeric,backend=bibtex,giveninits=true,sorting=nyt]{biblatex}
\addbibresource{references.bib}


% fromatovanie
\usepackage{indentfirst} % odsadenie prveho odstavca
\usepackage{enumitem} % číslovanie ((a),(b),(c),... i), ii), iii),...)
\usepackage[small]{caption} % popisy obrazkov


% grafika
\usepackage{graphicx} % obrazky
%\usepackage{tikz}
\usepackage{subcaption} % podobrazky (subfigure)
\graphicspath{ {./figures/} } % priecinok s obrazkami
\usepackage{wrapfig} % obrazky obtekane textom
\usepackage{xcolor} % farby
\usepackage{colortbl} % farebne tabulky
\usepackage{multirow} % treba pre merge-ovanie budniek v tabulkach

% matematika
\usepackage{mathtools} % rozsirenie zakladneho matematickeho balika amsmath
\usepackage{amssymb} % dalsie symboly a fonty (napriklad pre mnozinu realnych cisel)
\usepackage[locale=DE]{siunitx} % jednotky SI

% Definície, Vety, Dôkazy...
\usepackage{amsthm}
\theoremstyle{definition}
\newtheorem{thm}{Veta}[section]
\newtheorem{defn}[thm]{Definícia}
\newtheorem{lem}[thm]{Lemma}
\newtheorem{cor}[thm]{Dôsledok}
\newtheorem{rem}[thm]{Poznámka}
\newtheorem{exmp}[thm]{Príklad}


% hyperreferencia
\usepackage[hidelinks]{hyperref} % hyperreferencia
